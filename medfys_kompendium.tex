\documentclass[jmp, amsmath, amssymb, reprint]{article}

\usepackage[utf8]{inputenc}
\usepackage[english]{babel}
\usepackage{amsmath,graphicx,varioref,verbatim,amsfonts,geometry,grffile}
\usepackage[usenames,dvipsnames,svgnames,table]{xcolor}
\usepackage[colorlinks]{hyperref}
\usepackage{flafter}
\usepackage{float}
\usepackage{placeins}
\usepackage{fancyvrb}
\usepackage{comment}
\usepackage{blindtext}
\usepackage{enumitem}
\usepackage{subcaption}
% Document formatting
\setlength{\parindent}{0mm}
\setlength{\parskip}{1.5mm}
%Color scheme for listings
\usepackage{textcomp}
\definecolor{listinggray}{gray}{0.9}
\definecolor{lbcolor}{rgb}{0.9,0.9,0.9}
%Listings configuration
\usepackage{listings}
%Hvis du bruker noe annet enn python, endre det her for å få riktig highlighting.
\lstset{
	backgroundcolor=\color{lbcolor},
	tabsize=4,
	rulecolor=,
	language=python,
        basicstyle=\scriptsize,
        upquote=true,
        aboveskip={1.5\baselineskip},
        columns=fixed,
	numbers=left,
        showstringspaces=false,
        extendedchars=true,
        breaklines=true,
        prebreak = \raisebox{0ex}[0ex][0ex]{\ensuremath{\hookleftarrow}},
        frame=single,
        showtabs=false,
        showspaces=false,
        showstringspaces=false,
        identifierstyle=\ttfamily,
        keywordstyle=\color[rgb]{0,0,1},
        commentstyle=\color[rgb]{0.133,0.545,0.133},
        stringstyle=\color[rgb]{0.627,0.126,0.941}
        }
        
\newcounter{subproject}
\renewcommand{\thesubproject}{\alph{subproject}}
\newenvironment{subproj}{
\begin{description}
\item[\refstepcounter{subproject}(\thesubproject)]
}{\end{description}}

\lstset{inputpath="C:/Users/Jon Andre/Python/FYS2150"}
\graphicspath{{C:/Users/Jon Andre/Python/FYS2150/Magnetisme/}}
\numberwithin{equation}{section}
\usepackage{dcolumn}% Align table columns on decimal point
\usepackage{bm}% bold math
\usepackage{multicol}
\usepackage{physics}

\newcommand{\e}{\mathrm{e}}
\newcommand{\lp}{\left(}
\newcommand{\rp}{\right)}


\begin{document}

\title{Medical and Biological Physics}

\author{Jon A Ottesen}
%{Institute of physics, University in Oslo}%Lines break automatically or can be 
\title{Compendium FYS3700}% Force line breaks with \\
\date{\today}
\maketitle

\newpage
\tableofcontents
\newpage

\begin{multicols}{2}

\section{Atomic and Molecular physics}

\subsection*{Introduction}

Add introduction

\subsection{Quantum Mechanics}

At macroscopic scale QM becomes absolute but when the scale becomes microscopic QM is necessary to completely describe a system of atoms or molecules. Some solutions for the system is acquired by solving the Hamiltonian operator
\begin{equation}
 \hat{H} = \frac{\hat{p}^2}{2m} + V(\vec{r},t)
\end{equation}
one way or another, this is also known as the time independent Schrödinger equation. Sadly the only exactly solvable atom or molecule is the Hydrogen atom. For all other atoms or molecules consisting of more than a two-atomic system the electron repulsion makes the system \textbf{NOT} exactly solvable and approximations are necessary.

\subsubsection{Hydrogen}\label{sec:hydrogen}

As mentioned earlier the only exactly solvable atomic system is the Hydrogen atom. The solution to the hydrogen atom is given by the wave function\footnote{In QM a wave function is the complete mathematical description of a given system.}
\begin{equation}\label{eq:01}
\Psi\lp n,l,m_l, m_s\rp.
\end{equation}
This means that the complete description of the hydrogen atom depends on four distinct variables all describing different attributes of the atom. Each different combination of these numbers corresponds to a specific allowed state that the Hydrogen atom can take.

From Bohr's description of the hydrogen atom you may recognize the quantum number n. This quantum number indicates which quantized energy eigenvalue corresponds to the given state in the system. This quantized energy is given by
\begin{equation}
E_n = \frac{E_1}{n^2}\quad n=1,2,3,...
\end{equation}
where \(E_1\) is the lowest quantized energy, an important note is that n is only integers which in Bohr's atom model corresponds to a specific shell. Each of these shells are given a name starting at K(\(n=1\)), L(\(n=2\)) etc. Each shell can also hold up to \(2n^2\) electrons, the reason for this will be revealed in the orbitals section.

The second quantum number is l which describes the angular momentum of the system. Again as with the energy states this quantum number also has distinct eigenvalues
\begin{equation}
L = \hbar\sqrt{l\lp l+1\rp} \quad l=0, 1, 2,..., n-1.
\end{equation}
The third quantum number is the magnetic quantum number and gives the value of the angular momentum in the z-direction
\begin{equation}
L_z = \hbar m \quad m=-l, -l+1, ..., 0, ...,l-1, l.
\end{equation}
The last quantum number is $m_s$ and is the spin quantum number of the electron and takes the following values
\begin{equation}
m_s=\pm\frac{1}{2}.
\end{equation}

This might not seem like much but we have now found out that the complete description of the Hydrogen atom\footnote{I know i haven't given you the actual wave function but for out purpose it's unnecessary} only depends on four variables. We have also found both the permitted energy eigenvalues and angular momentum eigenvalues.

So far I haven't really said how the electron configuration is in the Hydrogen atom (or anything really), but I have said that a state of the Hydrogen atom is given by equation \ref{eq:01}. So what is really a state, it's just a possible placement\footnote{Placement is not really the correct word, after all it's really just a probability density.} for an electron in the Hydrogen atom. 

If you have looked closely at the permitted values for the quantum numbers you may have noticed that there are multiple states for each energy level. When multiple different states gives the same eigenvalue we have degeneracy. This degeneracy corresponds to different states an electron can have with said energy level or shell if you will. Meaning that there are multiple possible placements for an electron in each energy level or shell. This is important when describing the amount of electrons an atom can take in it's outer shell.

\subsubsection{Orbitals}

A orbital is a designation of the combination of quantum numbers that gives the state of the electron. These orbitals are categorized after which n and l value the state is in, the shape is also determined by the angular momentum quantum number l. Some orbital names are shown in table \ref{tabel:1}. By counting the different states in each orbital lets say 2p you find that there are a total of $m_s=-1, 0, 1$ magnetic states each with two different spins and therefore a total of 6 states in the 2p orbital. Each orbital does not only have different amounts possible electron states but the probability density also vary greatly as seen in figure \ref{fig:orbitals}.

\begin{table}[H]
  \begin{center}
    \begin{tabular}{| l | l | l | l | l |}
   	\hline
	 & \(l=0\) & \(l=1\) & \(l=2\) & \(l=3\)\\ \hline
	\(n=1\) & 1s &  &  & \\
	\(n=2\) & 2s & 2p &  & \\
	\(n=3\) & 3s & 3p & 3d & \\
	\(n=4\) & 4s & 4p & 4d & 4f\\ \hline
	\end{tabular}
    \caption{The different orbitals for each energy level and angular momentum.}
    \label{tabel:1}
  \end{center}
\end{table}
\FloatBarrier

\begin{figure}[H]
	\centering
  	\includegraphics[width=0.55\textwidth]{orbitals.png}
	\caption{The different shape of the s and p orbitals, the direction is determined by the magnetic quantum number \(m_l\).}
	\label{fig:orbitals}
\end{figure}

In the 2p orbital you already know that there are a total of 6 possible states or different electron combinations but does this mean that the orbital 2p can maximum accept 6 electrons? Can electrons share the same quantum numbers? The answer is \textbf{NO}. This comes from The Pauli principle: \textit{Two electrons cannot share all four quantum numbers}. This is important, let's for a moment imagine two hydrogen atom binding together. The Pauli principle than states that the electrons in this molecule cannot share the same quantum number but they can be degenerate (different spin) within the same \(l\) and \(m_l\) values. 

So far i haven't stated where electron prefer to be located. I won't elaborate much on this but they follow the principle of minimum energy, meaning that the electrons seek their lowest possible energy. In the \(H_2\) case this is the 1s orbital, this also means that the electrons fill up the orbitals with the minimum energy first\footnote{I won't elaborate much on this, but for us this means that electrons will fill up the orbitals 1s, 2s, 2p, 3s, 3p. The rest is not ordered as you would think from table \ref{tabel:1}, see internet.} in a given molecule. If you f.eks were asked to give the orbital configuration of Aluminum it would be: \(1s^2, 2s^2, 2p^6, 3s^2\) and \(3p^1\).

So far you know that that the orbitals are filled from the lowest energy and upwards but not how each orbital is filled. Another way of phrasing this is; which combination of the magnetic $m_l$ and spin $m_s$ quantum numbers would be used firstly to fill a orbital. To solve this we have Hund's rule: \textit{Electrons fill their states (orbitals) with as many parallel spins as possible}. In simpler terms this means that the 2p orbital would first fill all the $m_l$ values with spin \(m_s=\frac{1}{2}\) before filling up any \(m_l\) with spin with \(m_s=-\frac{1}{2}\)\footnote{When I say that \(m_s=\frac{1}{2}\) is filled up firstly this is just a convention that spin up is filled firstly.}. An illustration of this is shown in figure \ref{fig:orbitals}.

\begin{figure}[H]
	\centering
  	\includegraphics[width=0.50\textwidth]{hunds_rule.png}
	\caption{An illustration of Hund's rule.}%https://ch301.cm.utexas.edu/svg/hunds-rule.svg
	\label{fig:orbitals}
\end{figure}

\subsection{Atomic and molecular bonds}

\subsubsection{Chemical bonds}

For most of you this subsection might seem redundant since it will only cover the very basics but bonds are such a fundamental part of all macroscopic matter so it's absolutely a necessity.

Chemical bonds are often categorized by the strength of the bond. The bonds where there are needed large amount of energy the break the bond is considered strong while the weak bonds require less. The strong bonds are bonds that creates molecular structures through covalent bonds or crystal structure through ionic bonds. Weaker bonds are to weak to connect atoms together and instead work between molecules and create a more macroscopic structures. Weaker bonds are divided into polar, hydrogen and van der Waals bonds. To get a feel for the different distances and strength values see table \ref{tabel:2}.

\begin{table}[H]
  \begin{center}
    \begin{tabular}{| l | l | l |}
   	\hline
	Bonding type & length [nm] & Strength [kcal/mol] \\ \hline
	Covalent & 0.15 & 90\\
	Ionic & 0.25 & 3\\
	Hydrogen & 0.30 & 1\\
	van der Waals & 0.35 & 0.1\\ \hline
	\end{tabular}
    \caption{The strength and distance for the different types of bonds in.}
    \label{tabel:2}
  \end{center}
\end{table}
\FloatBarrier

Covalent bonds are bonds between atoms or between molecules and atoms forming molecules. When two atoms are bonded together with at covalent bond they share an electron pair and the repulsive and attractive forces are stable. This happens to the atoms when their electronegativity\footnote{'Electronegativity is a measure of the tendency of an atom to attract a bonding pair of electrons', try searching for a table you might find some pattern with where the electronegativity is largest and smallest.} is identical or relatively close. The physical reason for creating these bonds are to minimize the total energy of the system. Meaning that the total energy for a bonded molecule is less than if the atoms were roaming free without a full outer shell. Covalent bonds are again divided into two types of bonds sigma and pi bonds which we will discuss again later in the hybridization subsection.

Ionic bonds unlike covalent bonds doesn't share their respective electrons between atoms but instead transfer\footnote{Completely transfer of electrons in ionic bonds doesn't exist but the electronegativity will force the electron closer to one atom.} electrons to each other. This means that this bonding type require both a transfer atom and a receiver atom for the electron(s). This happens thanks to a large difference in electronegativity larger than \(1.7\) between the atoms, and we therefore have Coulomb interactions which stabilizes the structure. Much like covalent bonds this bond does also minimize the total energy of the system\footnote{When saying system I usually refer to a general system of a couple of atoms.}. Ionic bonds unlike covalent bonds can create macroscopic structures in the form of crystals like table salt NaCl. These structures have a high smelting point thanks to the high bond strength.

Polar bonds are a type of bonds which happens without any transfer or sharing of electrons. It's actually a weaker version of ionic bonds where the electrons isn't transferred. More specifically polar bonds occur when the electronegativity is smaller than \(1.7\). Polar bonds therefore often occur between molecules where the charge of the electrons are unevenly distributed between the atoms thanks the the difference in electronegativity in the atom. The polar part of the molecule is designated by the letter \(\delta^+\) while the negative is by \(\delta^-\). Most macroscopic structures are made up of molecules bonded together by the polar bonds between the molecules. A example of a macroscopic structure is water \(H_2O\), without the polar bonds there would be no water. A good analogy for polar bonds is to think of the molecules as microscopic magnets that attract each other and then forms a macroscopic structure thanks to the attractive forces. 

Hydrogen bonds are a \textbf{VERY} important sub-type of polar bonds often categorized as it's own type of chemical bond. The very reason being that hydrogen bonds are present in many of the most important structures to humans for example water. Hydrogen bonds occur as with polar bonds when there is a substantial difference between the electronegativity of a hydrogen in a molecule and another atom. This creates a \(\delta^+\) and a \(\delta^-\) making it possible to have electromagnetic forces keeping other types of polar molecules together. Example with a water (\(H_2O\)): The electronegativity of oxygen is 3.5 while hydrogen has 2.1. The oxygen will therefore have a greater pull on the electrons and it will be negatively charged compared to the hydrogen atoms. This results in two positively charges hydrogen atoms and one negatively charged oxygen atom. This results in a hydrogen bond between hydrogen and oxygen from different molecules thanks to Coulomb interactions. The hydrogen bond in water is depicted in figure \ref{fig:hydrogen_bond}.

\begin{figure}[H]
	\centering
  	\includegraphics[width=0.50\textwidth]{hydrogen_bond.png}
	\caption{An illustration of Hund's rule.}%https://commons.wikimedia.org/wiki/File:Hydrogen-bonding-in-water-2D.png
	\label{fig:hydrogen_bond}
\end{figure}

Van der Waals bonds are a special kind of snowflake when it comes to binding. They are clearly the weakest type of bonds but are still important when maintaining the structure when the molecules are non-polar. An example of non polar molecules are multiple gasses in the air \(O_2\), \(CO_2\) etc. Van der Waals bonds are weak interactions between molecules without any notable permanent dipolar moment\footnote{Dipolar moment is another way of defining the difference between the electronegativity within a molecule.}. So what are these weak interactions? From earlier QM courses you should know that the positions of electrons are given as probability densities located around the the atomic nucleus. This also means that the electrons aren't at a fixed location. A result of this is that the electrons can be heavily located at a certain section within the molecule. A high density of electrons in a specific area creates a net negative charge at a fixed position and as a result a net positive charge is created another place in the molecule. We therefore have a temporary dipole moment, which is used to form bonds with other temporary dipole moments. A general way of defining van der Waals are that they are created by temporary fluctuations in the charge distribution.

\subsubsection{Hybridization}

''Hybridization is Man’s way of mathematically being able to describe Mother Nature’s inherent need for a minimalization of the total energy for a given system.''

Hybridization is the concept of mixing atomic orbitals. A general definition of hybridization of orbitals is that they are a linear combination of the x states involved in the molecular structure, and the result is x new linearly independent, orthogonal states. The QM way of describing this would be
\begin{equation}\label{eq:02}
\Psi = \sum_{i=0}^N c_i\psi_i
\end{equation}
where \(\Psi\) is the new wave function for the molecule built up of N weighted \(c_i\) \(\psi\) states from other states. It's important to state that this is just an approximation since it's impossible to completely solve a more than two atomic system. Still this approximation closely resemble that of mother nature. So it's a model, but a good one!

I will now for the 'simple' \(H_2\) molecule use QM and hybridization to showcase the basics of hybridization. I firstly need to consider two hydrogen atoms A and B both in the s orbital that bonds together to the \(H_2\) molecule. Form hybridization this is then a linear combination of the states of A and B, from equation \ref{eq:02} this is normalized
\begin{align}
\Psi_{AB+} = d_{AB}\lp c_A\psi_A + c_B\psi_B\rp\label{eq:03}\\
\Psi_{AB-} = d_{AB}\lp c_A\psi_A - c_B\psi_B\rp.\label{eq:04}
\end{align}
An important note is that in equation \ref{eq:02} I don't exclude negativity as seen in the linear combination above. An illustration is shown in figure \ref{fig:wave_sum} where you can see how the electron density behaves depending on the sign in the wave function. 

\begin{figure}[H]
	\centering
	\begin{subfigure}{0.49\textwidth}
  	\includegraphics[width=1\textwidth]{h2_pos.png}
	\caption{Equation \ref{eq:03}.}
	\end{subfigure}
	\begin{subfigure}{0.49\textwidth}
  	\includegraphics[width=1\textwidth]{h2_neg.png}
	\caption{Equation \ref{eq:04}.}
	\end{subfigure}
	\caption{An illustration on how the electron density behaves at different signs.}
	\label{fig:wave_sum}
\end{figure}

There is a couple points of importance from figure \ref{fig:wave_sum} and thereby with equation \ref{eq:03} and \ref{eq:04}. Equation \ref{eq:03} gives high electron density in the middle of the nuclei and the atoms share bonding orbitals. In equation \ref{eq:04} the electron density is lower and there are no shared orbitals. This leads to two different types of bonds between the atoms, the regular bond which belongs to equation \ref{eq:03} and one anti-bond which belongs to equation \ref{eq:04}. Regular bonds are more stable and have a low internal energy while anti-bonds are unstable and their energy is higher than when separated. The energy differences can be seen in figure \ref{fig:bond_energy}. It's important to know that anti-bonding and bonding isn't classification on the type of covalent bonds, but rather the composition of the wave function. A general way of though is that negative signs corresponds to repulsion of the electrons densities in regards to one another. A mathematical proof for the difference in energy between anti and regular bonds can be found in...

\begin{figure}[H]
	\centering
  	\includegraphics[width=0.50\textwidth]{anti_bond_energy.png}
	\caption{A illustration of the energy difference between anti-bonds and regular bonds.}
	\label{fig:bond_energy}
\end{figure}

Covalent bonds are separated into two different types: \(\sigma\) and \(\pi\) bonds. Both of these bonding types occure both as anti and regular bonds, meaning you can have an anti \(\sigma\) bond. These bonds are separated by how the orbitals are situated. Sigma bonds are bonds created by overlapping of orbitals. The definition is that: \(\sigma\)-bonds are bonds where the orbitals are symmetric with respect to the molecular axis\footnote{The axis between the nuclei the bond is formed.}. This means that sigma bonds can be created by both a mixture of s, p, d etc orbitals as long as the linear combination is symmetric, this is shown in figure \ref{fig:sigma_bonds}. A \(\pi\)-bond is a bond defined as: A \(\pi\)-bond is asymmetric with respect to the nodal plane\footnote{The nodal plane is a plane where the electron density is zero, in this case on the molecular axis.}. \(\pi\) bonds are weaker than \(\sigma\)-bonds since their overlapping of orbitals are less prominent. A example of \(\pi\)-bonds are shown in figure \ref{fig:pi_bonds}. In the figure I have also shown anti \(\pi\)-bonds, these are asymmetric around both axes shown in the figure, the difference in sign on the figure is a result of the sign difference in the wave function.


\begin{figure}[H]
	\centering
  	\includegraphics[width=0.50\textwidth]{sigma_bonds.png}
	\caption{\(\sigma\)-bonds between diffrent orbitals.}
	\label{fig:sigma_bonds}
\end{figure}

\begin{figure}[H]
	\centering
  	\includegraphics[width=0.50\textwidth]{pi_bonds.png}
	\caption{Both \(\pi\)-bonds and anti \(\pi\)-bonds.}
	\label{fig:pi_bonds}
\end{figure}

\subsubsection{sp - sp\(^2\) - sp\(^3\) hybridization}

We will here look at three different hybridization types that are essential in organic material. When finding the hybrid orbital made up of different orbitals the new hybrid orbital will have new energy that is between that of the originals as illustrated in figure \ref{fig:hybrid_energy}. When a hybrid orbital is created on a atom that is bonding with other atoms the hybrid orbital is a linear combination of different states on the bonding atom.

\begin{figure}[H]
	\centering
  	\includegraphics[width=0.50\textwidth]{hybrid_orbital_energy.png}%https://www.chemicool.com/img1/graphics/sp3energylevels1.png
	\caption{Illustration on how the energy is for a hybrid orbital compared to that of the originals.}
	\label{fig:hybrid_energy}
\end{figure}

Sp hybridization is a type of hybridization that occurs when combing the 2s orbital with one 2p orbital. This results in two new sp hybrid orbitals while the atom keeps two p orbitals. The wave functions for sp orbitals are
\begin{align}
\Psi_1 = \frac{1}{\sqrt{2}}\lp \psi_{2S}+\psi_{2px}\rp\\
\Psi_2 = \frac{1}{\sqrt{2}}\lp \psi_{2S}-\psi_{2px}\rp.
\end{align} 
The sp orbitals are located at the same axis while the two p-orbitals are standing perpendicular on this axis and on each other. An example on sp hybridization is shown in figure \ref{fig:sp_orbital}. As seen on the figure this produces two \(\pi\)-bonds between the p-orbitals on the carbon atoms. A general rule is that with triple bindings between carbon is that it's sp hybridized and therefore two of these bindings are \(\pi\)-bonds.

\begin{figure}[H]
	\centering
  	\includegraphics[width=0.50\textwidth]{sp_orbital.png}
	\caption{Sp hybridization on ethyne \(C_2H_2\).}
	\label{fig:sp_orbital}
\end{figure}

Sp\(^2\) hybridization is a type of hybridization that occurs when combing one 2s orbital with two 2p orbitals. This results in three new sp\(^2\) hybrid orbitals while one p orbital still remain. The wave functions for sp\(^2\) orbitals are
\begin{align}
\Psi_1 = \frac{1}{\sqrt{3}}\lp \psi_{2S}+\sqrt{2}\psi_{2px}\rp\\
\Psi_2 = \frac{1}{\sqrt{3}}\lp \psi_{2S} -\frac{1}{\sqrt{2}}\psi_{2px} + \frac{\sqrt{3}}{\sqrt{2}}\psi_{2py}\rp\\
\Psi_3 = \frac{1}{\sqrt{3}}\lp \psi_{2S} -\frac{1}{\sqrt{2}}\psi_{2px} + \frac{\sqrt{3}}{\sqrt{2}}\psi_{2py}\rp.
\end{align}
The sp\(^2\) orbitals are arranged in a triangular shape on the same plane with angles \(120^\circ\) between. The last p orbital is perpendicular to said plane. An example of a sp\(^2\) hybridized molecule is \(C_2H_4\) which is shown in figure \ref{fig:sp2_orbital}. The triangular shape of the sp\(^2\) orbitals are very important for the general shape of molecules as seen in the figure. A general rule is that for sp\(^2\) hybridization it's one \(\pi\) and one \(\sigma\) bond between the carbon atoms.

\begin{figure}[H]
	\centering
  	\includegraphics[width=0.50\textwidth]{sp2_orbital.png}%https://www.slideshare.net/pedagogics/2012-orbital-hybrization-sigma-and-pi-bonds
	\caption{Sp\(^2\) hybridization on ethene \(C_2H_4\).}
	\label{fig:sp2_orbital}
\end{figure}

Sp\(^3\) hybridization is a type of hybridization that occurs when combing one 2s orbital with three 2p orbitals. This results in four new sp\(^3\) hybrid orbitals with no remaining p orbitals. The wave functions for sp orbitals are
\begin{align}
\Psi_1 = \frac{1}{2}\lp \psi_{2s} + \psi_{2px} + \psi_{2py} + \psi_{2pz}\rp\\
\Psi_2 = \frac{1}{2}\lp \psi_{2s} + \psi_{2px} - \psi_{2py} - \psi_{2pz}\rp\\
\Psi_3 = \frac{1}{2}\lp \psi_{2s} - \psi_{2px} + \psi_{2py} - \psi_{2pz}\rp\\
\Psi_4 = \frac{1}{2}\lp \psi_{2s} - \psi_{2px} - \psi_{2py} + \psi_{2pz}\rp.
\end{align}
The sp\(^3\) orbitals are arranged in a tetrahedral shape on the same plane with a angel \(\approx 109^\circ\) between. An example of a sp\(^3\) hybridized molecule is \(C_2H_4\) which is shown in figure \ref{fig:sp3_orbital}. The tetrahedral shape of the sp\(^3\) orbitals are important for the general shape of molecules as seen in the figure. For sp\(^3\) there are only \(\sigma\)-bonds between the atoms.

\begin{figure}[H]
	\centering
  	\includegraphics[width=0.50\textwidth]{sp3_orbital.png}%https://i.pinimg.com/originals/bc/9b/e4/bc9be459f7c5bb02c204f35ca16fd2c0.jpg
	\caption{Sp\(^3\) hybridization on ethane \(C_2H_6\).}
	\label{fig:sp3_orbital}
\end{figure}

\clearpage

\section{Radioactivity}

\subsection*{Introduction}

Add introduction

\subsection{Radiation}

Radiation is the emission or transmission of energy in the form of waves or particles. The transmission of energy in waves are through electromagnetic radiation like: gamma, x-ray, ultraviolet, visible light etc. The energy of electromagnetic radiation is given by
\begin{equation}
E=h\nu \qquad \nu = \frac{c}{\lambda}.
\end{equation}
The transmission of energy through particles are typically from: alpha particles, electrons, positions, protons etc.

The energy transmitted through radiation is relatively small compared to energy measured on macroscopic systems. Therefore it's advantageous to change units to electron volt instead of the regular joule where
\begin{equation}
1\text{eV}=1.6\cdot 10^{-19}\text{joule}.
\end{equation}
The advantage of using eV can be seen in the rest energy of some particles in table \ref{tabel:3}. The rest energy seen in the table is given by
\begin{equation}
E_0=m_0c^2
\end{equation}
while the relativistic rest energy for a object in motion is
\begin{equation}
E_0=m_0c^2\qquad m=\dfrac{m}{1-\lp\frac{v}{c}\rp}.
\end{equation}
Lastly the total energy is given by
\begin{equation}
E^2=\lp pc\rp^2 + \lp m_oc^2\rp^2\qquad p=\frac{m}{\sqrt{1-\lp\frac{v}{c}\rp^2}}
\end{equation}

\begin{table}[H]
  \begin{center}
    \begin{tabular}{| l | l | l |}
   	\hline
	Particle & Rest mass [kg] & Rest energy [MeV]\\ \hline
	Electron & \(9.11\cdot10^{-31}\) & \(0.511\)\\
	Proton & \(1.67\cdot10^{-27}\) & \(938\)\\
	Neutron & \(1.67\cdot10^{-27}\) & \(938\)\\
	Alpha particle & \(6.64\cdot10^{-27}\) & \(3727\)\\ \hline
	\end{tabular}
    \caption{The different mass and rest energy of a couple of particles.}
    \label{tabel:3}
  \end{center}
\end{table}

From section \ref{sec:hydrogen} you should already know that the hydrogen atom has quantized energy states. The quantized trait is actually more general, and all atoms have quantized energy states. This is the reason for the shell categorization and the principle behind the Bohr atomic structure with the nucleus at the center surrounded by shells housing the electrons.

The fact that the energy of the electrons are quantized leads to many interesting consequences. The most important for us is the fact that the energy emitted thorough electromagnetic radiation is equal to the gap in energy between two shells. The electromagnetic transition between the shells in hydrogen and tungsten is shown in figure \ref{fig:el_transition}, it's not important to know anything specific for these two. But rather that the transitions are random from the n-th energy level down to the lowest possible energy state but the transition is always to a shell with lower energy. Whenever a electron transitions from a higher shell to a lower the energy difference is emitted through electromagnetic radiation.

\begin{figure}[H]
	\centering
  	\includegraphics[width=0.50\textwidth]{electromagnetic_transistion.png}
	\caption{The electromagnetic transitions between the shells in hydrogen and tungsten.}
	\label{fig:el_transition}
\end{figure}

\subsubsection{Radioactive decay}

No point in filling in. Should be pretty straight forward.

\subsection{Ionization}

Ionization is the process by which an atom or a molecule acquires a negative or positive charge by gaining or losing electrons to form ions. In our case when referring to ionization we mostly refer to the process when the atom/molecule gains a positive charge and therefore losses an electron.

Ionization can happen though radiation when the energy of the radiation (particles or EM) is greater or equal to the absolute energy of the electron with the maximum energy. Mathematically this relation is
\begin{equation}
0 \leq E_r + E_n
\end{equation}
where \(E_r\) is the energy of the radiation and \(E_n\) is the energy of the electron. A couple of these energy levels are shown in table \ref{tabel:4}. When ionizing an atom the total energy is conserved. Therefore the total energy before the ionization equals the energy of the ionized electron plus the energy of the ionizing particle or photon after the ionization, mathematically:
\begin{equation}
E_{in} = E_{e-} + E_{p/f}.
\end{equation}


\begin{table}[H]
  \begin{center}
    \begin{tabular}{| l | l |}
   	\hline
	Element & Ionization potential [eV]\\ \hline
	H & 13.6\\
	He & 24.5\\
	C & 11.3\\
	O & 13.6\\
	Mo & 7.1\\
	W & 7.9\\
	H\(_2\)O & 12.6\\ \hline
	\end{tabular}
    \caption{A couple of ionization potentials.}
    \label{tabel:4}
  \end{center}
\end{table}

The minimum energy required for ionization doesn't represent how easy an atom or molecule is ionized. Instead we use the mean excitation energy which measure the mean energy required for ionization of a certain material. Some mean excitation energies is given in table \ref{tabel:5}, notice how they are larger than the potentials given in table \ref{tabel:4}. The mean excitation potentials roughly scales with
\begin{equation}
\left<E_{mean}\right>\sim 10Z
\end{equation}
where Z is the atomic number.

\begin{table}[H]
  \begin{center}
    \begin{tabular}{| l | l |}
   	\hline
	Element & Mean excitation energy [eV]\\ \hline
	H & 19\\
	C & 81\\
	Pb & 823\\
	H\(_2\)O & 75\\ \hline
	\end{tabular}
    \caption{A couple of mean ionization potentials.}
    \label{tabel:5}
  \end{center}
\end{table}

\subsubsection{Excitation}

Not all radiation leads to ionization. Rather most radiation leads to a process called excitation. Excitation happens when the electromagnetic radiation equals the energy difference between a bound electron (an electron within a orbital in the atom or molecule) and a higher shell or orbital and hits the electron. When this happens the electron elevates to a higher shell or preferably called state. If you remember the principle of minimum energy you should also know that the electron prefers a state with the minimum energy. Therefore the electron goes through a process called de-excitation. When this happens the electron 'jumps' to a state with lower energy and the difference in energy is released in the form of electromagnetic radiation. The amount of states lowered towards it's minimum energy is random so multiple photons can be released during the de-excitation as stated in the beginning of the radiation section.

Useful quantities:
\begin{enumerate}
\item \(N_A\): Avogadro constant
\item \(A_r\): Atomic weight
\item \(Z\): Atomic number
\item \(\rho\): Density
\item 1 amu: \(1/12\) of the mass of a \(^{12}\)C atom.
\item \(N_{am}=\frac{N_A}{A_r}\): Atoms per unit mass
\item \(ZN_{am}=\frac{Z}{A_r}N_A\): Electrons per unit mass
\item \(ZN_{aV}=\rho ZN_{am}\): Electrons per unit volume.
\end{enumerate}

\subsubsection{Absorbed, equivalent and effective dose}

Absorbed dose is defined as the energy deposited per mass as
\begin{equation}
D=\frac{E}{m}
\end{equation}
with unit [Gy]=[J/kg]. It's more generally thought of as the amount of energy absorbed per kg of tissue. This is important when talking about the amount of radiation someone or something is receiving.

Equivalent dose is the biological effectiveness of the different types of radiation when exposed to tissue. This is given by the absorebed dose times a correction factor for a radiation type as
\begin{equation}
H_T = D\cdot w_R
\end{equation}
where \(w_R\) is the correction factor also known as radiation weighting factor, \(D_e\) is meassured in [Sv]-sivert. Multiple values for \(w_R\) for different types of radiation is given in table \ref{tabel:6}.

\begin{table}[H]
  \begin{center}
    \begin{tabular}{| l | l |}
   	\hline
	Radiation & \(w_R\) [unit-less]\\ \hline
	EM-radiation, beta, muons & 1\\
	Neutrons & 5-20\\
	Protons & 2\\
	Alpha, heavy nuclei & 20\\ \hline
	\end{tabular}
    \caption{Radiation weighting factor for different radiation types.}
    \label{tabel:6}
  \end{center}
\end{table}

Effective dose is a way of measuring the potential harm radiation has on different parts of the body. This is done through a organ weighting factor and the effective dose is given by
\begin{equation}
D_{ef}=D\cdot w_R\cdot w_T
\end{equation}
where \(w_T\) is the organ weighting factor, \(D_{ef}\) is also measured in [Sv]-sivert. It's important to know that the organ weighting factor is very uncertain.

When the whole body is irradiated the total equivalent and effective doses are just written as a sum over the equivalent and effective dose for the different radiation types and different regions of the body. This can be done as follows
\begin{equation}
H_T=\sum_R w_RD_R
\end{equation}
\begin{equation}
D_{ef}=\sum_Tw_TH_T=\sum w_T\sum_R w_RD_R.
\end{equation}

\subsection{Types of ionizing radiation}

Ionizing radiation is often categorized into directly and indirectly ionizing radiation based on the nature of the radiation. Directly radiation is fast charged particles that deposit their energy in matter directly. This is done with many small Coulomb interactions along the particles track. Indirectly radiation is electromagnetic waves or neutron that transfer their energy to charged particles released in direct interactions. The resulting charged particles then deposit their energy in direct interactions. An illustration for this is shown in figure \ref{fig:direct_indirect}

\begin{figure}[H]
	\centering
  	\includegraphics[width=0.50\textwidth]{direct_indirect.png}
	\caption{An illustration of direct and indirect radiation.}
	\label{fig:direct_indirect}
\end{figure}

Indirectly ionization is divided into three different processes called: photoelectric effect, Compton scattering and pair production.

\subsubsection{Photoelectric effect}

Photoelectric effect is the process which happens when a photon has energy larger than the absolute energy of an electron. When this happens and the photon hits the electron the electron is freed from the nucleus and becomes a free electron and the photon is totally absorbed. When this happens the kinetic energy of the electrons is given by
\begin{equation}
E_k =h\nu-|E_n|.
\end{equation}
This process occur most frequently with electrons with lower energy states. Further this process it more likely in heavier atoms and the probability increases with \(\sim Z^3\). Lastly this process is less likely for increasing photon energy \(\sim h\nu^{-3}\). Meaning that this process happens for low energy photons. Remember that since the photons hits electrons this process happens far outside the nucleus.

\subsubsection{Compton scattering}

Compton scattering is a process where a loosely bound electron is hit by a photon and results in a free electron and a photon with less energy than before. The process of Compton scattering was important when proving the conservation of momentum for photons. The change in wavelength for the photon is given by
\begin{equation}
\lambda_i-\lambda_f=\Delta\lambda=\frac{h}{m_ec}\lp 1-\cos\lp\theta\rp\rp.
\end{equation}
For the meaning of the different signs see figure \ref{fig:compton}. As you can see the result of Compton scattering is both a free electron and a photon. It therefore makes sense that this process happens when the incoming photon has more energy than that of photoelectric effect. The probability of Compton scattering scales with \(\sim Z\) and with \(\sim h\nu^{-1}\). This process is therefore more often seen for medium energy photons with elements with lower atomic numbers. Remember that since the photons hits electrons this process happens far outside the nucleus.

\begin{figure}[H]
	\centering
  	\includegraphics[width=0.50\textwidth]{compton.png}
	\caption{An illustration of Compton scattering.}
	\label{fig:compton}
\end{figure}


\subsubsection{Pair production}

Pair productions is a process that happens when the energy of the photon is larger than the rest mass energy of both an electron and positron\footnote{More generally it's actually for a particle and it's corresponding anti particle.}
\begin{equation}
E_{photon} > E_{e^-} + E_{e^+}.
\end{equation}
When this condition is met and a photon travels close to the nucleus it changes to a electron and positron. In this process the entire energy of the photon is transferred to the particles in a way that momentum is conserved. Since it's required a large amount of energy to create two particles this process happens for high energy photons. The probability of pair production scales with \(\sim Z^2\) and with \(\sim h\nu\). This process is therefore more often seen for high energy photons with elements with higher atomic numbers. Unlike photoelectric effect and Compton scattering this process happens close to the nucleus.

A figure representing the three types of photon interactions is shown in figure \ref{fig:photon_interactions}.

\begin{figure}[H]
	\centering
  	\includegraphics[width=0.50\textwidth]{photon_interactions.png}
	\caption{An illustration of the different types of photon interactions with matter.}
	\label{fig:photon_interactions}
\end{figure}

\subsubsection{Photon attenuation}

Before starting to talk about photon attenuation I want to introduce the inverse square law. This law states that the total amount of photons in a vacuum that passes through a sphere no matter the distance r from the source is constant. An important consequence of this is that the photon density decreases with
\begin{equation}
I\propto \frac{1}{r^2}.
\end{equation}

In regular cases photons isn't in vacuum. The definition on how the intensity of the photon beam changes due to interactions is given by
\begin{equation}
\frac{dI}{dx}=\mu I,
\end{equation}
which is just states that the change in intensity is given by a constant times the intensity. By doing the necessary integration the change in intesity is given by
\begin{equation}
I=I_0\e^{\mu x}.
\end{equation}
In this equation \(\mu\) is the attenuation coefficient and reflects the total probability of a photon interactions per unit length. If the half intensity thickness is know the coefficient can be calculated by setting \(I=I_0/2\) and solving with respect for \(\mu\).

It was stated earlier that the different types of photon interactions happens at different energy values. This can be seen on how the linear attenuation coefficient \(\mu\) changes depending on the energy. The effect diffrent energy levels has on the attenuation coefficient can clearly be seen in figure \ref{fig:lac}.

\begin{figure}[H]
	\centering
  	\includegraphics[width=0.50\textwidth]{lac.png}
	\caption{How the linear attenuation coefficient changes depending on the energy of a photon in iron.}
	\label{fig:lac}
\end{figure}

\subsection{Radiation on matter}

When photons are radiating tissue we can measure typical depth dose curves as shown in figure \ref{fig:depth_dose}. If you look clearly will see that the top on the curves are shifted to the right, this is because of the so called ''buildup zone''. The shift in maximum energy can be explained by the fact that free electrons cause a majority of ionization. Therefore in the beginning when the radiation from the photons are at it's maximum the most electrons will be ionized. When this happens the electron cause a major secondary ionization which is so great that the maximum dose is shifted.

\begin{figure}[H]
	\centering
  	\includegraphics[width=0.50\textwidth]{dept_dose.png}
	\caption{Depth dose curves for different energy levels of photon radiation in water.}
	\label{fig:depth_dose}
\end{figure}

\subsubsection{Bethe's stopping power}

As said earlier charged particles are continuously interacting with matter through Coulomb interactions and thereby slowing down. This slowing down is descirbed in Bethe's stopping formula which determines the change in energy per pathlength. This formula is expressed by the particle speed (v), charge (z), electron density (n) and the mean exciation energy for the medium (I) as
\begin{equation}\label{eq:05}
S=\frac{dE}{dx}=\frac{4\pi nz^2}{m_ev^2}\lp\frac{e^2}{4\pi\epsilon_0}\rp^2\ln(\frac{2m_ev^2}{I}).
\end{equation}
A plot of the stopping power versus the energy E is shown in figure \ref{fig:bethes_energy}. A way more interesting and important plot from Bethe's formula is shown in figure \ref{fig:bragg} which is a plot of a Bragg curve. I want you to specifically notice the top called a Bragg peak. On the Bragg peak the stopping power is at it's greatest meaning that on this length the most energy is deposited. This is because at this length the particle has lower speed compared to before and therefore has more time to interact with matter. Also think of the applications if the top represented a tumor.


\begin{figure}[H]
	\centering
  	\includegraphics[width=0.50\textwidth]{bethe_energy.png}
	\caption{Bethe's stopping power with exsperimental data plot.}
	\label{fig:bethes_energy}
\end{figure}

\begin{figure}[H]
	\centering
  	\includegraphics[width=0.50\textwidth]{bragg.png}
	\caption{Bragg curve.}
	\label{fig:bragg}
\end{figure}

Stopping power S reflects the ionization density of the tissue for the used particle. This means that the larger the stopping power the more ionizations are present. Because of this the stopping power is sometimes (inaccurately) referred to as \textit{linear energy transfer - LET}. Also the greater the stopping power the shorter the particle will travel in the tissue.

From figure \ref{fig:bragg} you could clearly see a distinct peak called bragg peak. This peak can be made shallower with different types of particle radiation. Proton is a type of particle with a clearly defined top, but heavier ions generally gives even more distinct tops. The problem is that heavier ions continue to ionizing little after the top is reached instead of falling to zero. This is because heavier ions have a tendency to break apart, and since their new mass and charge is smaller they travel further.

It's possible to extend this peak over a larger area without increasing the ionization before or after the peak. This is done by changing the start velocity of the protons used and thereby shift each individual peak and creating a continues top.


\subsubsection{Scattering}

Bethe's stopping power is important when studying the depth a particles travels within a tissue but it doesn't tell anything about the path the particle travels or about how the interacted particles react. Earlier I stated that particles are the type of direct radiation and therefore are interacting through coulomb interactions. An important consequence of this is that the closer the particle is to the nucleus of an atom the greater is the Coulomb force which acts radially between the particles. For small particles like electrons this force will lead to deflected tracks. Meaning that the trajectory won't be straight, the lesser the mass of the radiated particle the larger the deflection. An example of the difference in deflection is shown in figure \ref{fig:scattering}. I also want to mention the side track in the \(\alpha\) trajectory is an ionized electron. Tese types of side chains cause by ionized electron are also called \(\delta\)-electrons.

\begin{figure}[H]
	\centering
  	\includegraphics[width=0.50\textwidth]{scattering.png}
	\caption{The difference in deflection between electrons and \(\alpha\)-particles.}
	\label{fig:scattering}
\end{figure}

\subsubsection{Bremsstrahlung}

As stated in the previous subsection electrons are deflected greatly when they pass close to the nucleus. When this happens a \(\gamma\)-quant is emitted. The energy emitted by the \(\gamma\)-quant is equal to the energy lost by the deflected particle when decelerating. This is not important for heavy particles since their deflections is close to negligible in comparison to that of an electron.

\newpage

\section{Cell biology}

\subsection*{Introduction}

The understanding of the cellular structure and the mutual relationships between cells is essential for the understanding of biological phenomenon. To understand the cell it's necessary to understand its complex internal structure and how the different parts are connected. 

\subsection{The principle of reductionism}

Scientific reductionism is the idea of reducing complex interactions and entities to the sum of their constituent parts, in order to make them easier to study. This is ofcourse often not possible in biology and it's necessary to look at the system as a whole rather than at its individual parts.

\subsection{History of the cell}

For around 4 billion years ago the atmosphere was vastly different from today. There were little to no free oxygen instead mostly consisting of \(CO_2\) and \(N_2\), such an atmosphere provides reduced potential for the probability of forming organic molecules from sunlight or electric sparks. With the combination of \(H_2\), \(CH_4\) and \(NH_3\) in the proximity of water vapor led through the formation of organic molecules when there were electrical discharges. As time passed these organic molecules evolved into primitive cellular structure with genetic material freely floating within an enclosed membrane. These primitive cellular structures were the first cells being called prokaryotic and were anaerobic.

As time passed the oxygen levels increased thanks to the first primitive cells which could preform photosynthesis. When the oxygen level increased a new aerobic cells were formed around 2 billion years ago with a more complex inner structure consisting of several organelles/organs. Some of these organelles are earlier prokaryotic cells absorbed into the new eukaryotic cell. The mitocondira is a example of an earlier prokaryotic cell absorbed into the eukaryotic cell. Unlike the earlier cells, eukaryotic cells use oxygen for energy production.

Cells are therefore divided into two groups: prokaryotic and eukaryotic cells where the difference is the complexity of their cellular structure.

\subsection{Tissue}

The human body consists of multiple types of cells with a variety of different tasks. Because of this variety we have classified the into five distinct groups being epithelial tissue, connective tissue, muscle tissue, nervous tissue and blood cells. All of these types are again divided into several subgroups but these are not important for us.

\subsubsection{Epithelial tissue}

Epithelial tissues line the outer surfaces of organs and blood vessels, as well as the inner surfaces of cavities in many internal organs. A great example of epithelial tissue is the human skin.

The epithelial has multiple important functions. These include the the protection of tissue and organs from outer stimuli as bacteria. It acts as a barrier between different regions with different constituents. Further an important function is selective transport and exchange of chemicals between the layers of epithelial tissue. And lastly is an important sensing organ.

\subsubsection{Connective tissue}

Connective tissue is used in the body to maintain form and support throughout the inner and outer body. The connective tissue consist of reticular connective which is mostly fibroblasts called reticular cells and acts as scaffolding for other cells in several organs maintaining its structure. We also have adipose or fat cells, these acts as energy storage, protective padding around the organs and isolation. We also have bones and cartilage which both acts as the supportive system of the entire body keeping it stable and supports the body. Further the 'ground substance' of the extracellular matrix is an amorphous gelatinous material. It is transparent, colorless, and fills the spaces between fibers and cells. And lastly different types of white blood cells.

\subsubsection{Muscle cells}

Muscle cells are the cells within the body that generate mechanical forces and movement. These are divided into smaller groups. Firstly we have the skeletal muscle cells which are voluntary muscles and are the most common muscle group. This muscle type are made of long fibers with multiple cell nucleus where the fibers are connected to the skeleton structure and have strong and fast contractions. The second is the smooth muscle cells and they are involuntary. They like skeletal consist of long cells but with one singular nucleus. They are associated with the  oesophagus, stomach, intestinals and blood vessels. More general it constitutes much of the musculature of internal organs and the digestive system with slow and automatic movements. Lastly we have the cardiac muscle cells also called muscle cells and they are also involuntary. They are the muscles keeping the heart pumping thanks to the electrical conductivity regulating the timing of the heartbeats keeping them stable.

\subsubsection{Nervous tissue}

Nervous tissue is a type of cell transmitting electrical signals throughout the body. The major function of the nervous tissue is to form a communication network by conduction electrical currents. The nervous cells have a regular cell body with nucleus and organelles but also multiple distinct structures. The dendrites are branch like structure coming out of the cell and they catch input signals to the neuronal cell\footnote{This is just a more fancy way of saying nervous cell, I prefer this one.} and generate electrical responses. The axon it the conductor of the electrical signal from the neuronal cell to the end station and can become over 1 meter long. These axons are often covered in myelin sheaths, which increases the speed of the impulse transmission. Lastly we have the axon terminal and the synapses. The axon structure transfer the signal from one neuronal cell to another cell (often another neuronal cell) by the synapse which is the contact between the axon terminal and the other cell body.

\subsubsection{Blood cells}

The human blood consist of some major types of blood cells being: red blood cells, white blood cells and platelets, but also something called plasma (55\% which is mainly water but includes many important substances as proteins). Red blood cells are cells with no nucleus and are the most common type found in the human body and are ca 1000 times more abundant then white blood cells. Their major function is to transport \(O_2\) and \(CO_2\) throughout the body. They are extremely flexible and often assume a bell shape to pass through the blood vessels. The second is white blood cells and are important to the immunity system (see immunology). The last type is blood platelets and have no cell nucleus and coagulate blood.

\subsubsection{Cell differentiation}

Cellular differentiation is the process by which a less specialized cell becomes a more specialized cell type to carry out distinct functions. The process of cell differentiation is driven by genetics and their interaction with the environment. 

Every single somatic cell in your body contain the exact same DNA. Cell differentiation is the process to permanently\footnote{At least in humans} turn off/on different genes (protein recipes) and thereby specializing the protein production and the cell type. Different protein productions change the look and functionality of the cell. The way cells determine which genes to turn off/on is determined by external stimuli also known as cues. An important note is that it's a hierarchy of cell differentiation, meaning that a stem cell can turn into all cells but a muscle cell can only turn into types of muscle cell \textbf{NOT} other types. An example of cell differentiation for blood cells are shown in figure \ref{fig:cell_differentiation}.

\begin{figure}[H]
	\centering
  	\includegraphics[width=0.50\textwidth]{cell_differentiation}%https://www.cancer.gov/images/cdr/Live/CDR526538.jpg
	\caption{Cell differentiation for blood cells.}
	\label{fig:cell_differentiation}
\end{figure}

\end{multicols}

\subsection{Eukaryotic cell}

\begin{figure}[H]
	\centering
  	\includegraphics[width=0.8\textwidth]{cell_eu}
	\caption{A eukaryotic cell.}
	\label{fig:cell_eu}
\end{figure}

\begin{multicols}{2}

\subsubsection{Cell membrane}

Cell membrane is the membrane separating the cellular interior and its surroundings. The cellular membrane is encased in a membrane made out of lipids, proteins and carbohydrate regulating the interaction with the environment. This can be seen through the intake an excrete of various substances. An illustration of the cellular membrane is shown in figure \ref{fig:cell_membrane}.

\begin{figure}[H]
	\centering
  	\includegraphics[width=0.5\textwidth]{cell_membrane.png}%https://upload.wikimedia.org/wikipedia/commons/d/da/Cell_membrane_detailed_diagram_en.svg
	\caption{An illustration of the cellular membrane.}
	\label{fig:cell_membrane}
\end{figure}

The most common substance in the cellular membrane is lipids, not only in the outer cellular membrane but also around the organelles. Lipids are bipolar molecules\footnote{Yes, just like SKAM Even.} meaning that the molecule have both a polar section and a non-polar section. In lipids the 'head' is the polar part of the molecule and therefore the it's hydrophilic since it can create polar bonds to water. The 'tail' is non-polar and therefore hydrophobic since it cannot create polar bonds to water. The majority of the lipids in the cellular membrane consist of phospholipids shown in figure \ref{fig:lipid_phos}. The upper part is the 'head' while the lower part is the 'tail'. The bipolar nature of the membrane leads to a bilayer structure which consist of two singular layers as seen on figure \ref{fig:cell_membrane}, the distance between the layers are often around 60-80 Å.

The two other major types of lipids are glycolipids and sterols, the amount depends on the cell type. The fatty acid chains phospholipids and glycolipds often contain an even number of carbon atoms typically between 16-20. They may also be saturated (no double bonds between the carbons) or unsaturated (one or more double bonds between the carbons). The length and saturation of these chains determine the fluidity of the membrane, the less compact the acids become the greater is the fluidity.

The cellular membrane is held together by non-covalent\footnote{A non-covalent interaction differs from a covalent bond in that it does not involve the sharing of electrons,[1] but rather involves more dispersed variations of electromagnetic interactions between molecules or within a molecule.} interactions, however the structure is fluid and not fixed rigid in place. Under physiological conditions (regular conditions) the lipid molecules are free to exhibit rapid lateral movement along the layer in which they are present. However movement between layers in a bilayer structure is very slow in comparison.

\begin{figure}[H]
	\centering
  	\includegraphics[width=0.5\textwidth]{lipid_phos.png}%http://www.madsci.org/posts/archives/2006-12/1164999854.Bc.1.gif
	\caption{A phospholipid molecule, the most abundant molecule in the cellular membrane.}
	\label{fig:lipid_phos}
\end{figure}

The cellular membrane also consists of carbohydrates. One of the most important functions of carbohydrates on the cellular membrane is the cell-cell recognition (information sharing etc) in eukryotes.

Lastly in the structure we have proteins which typically consist of 50\% of the membrane volume. These proteins are important because of their various biological activities. Membrane proteins consist of three main types: Integral proteins, peripheral proteins, and lipid-anchored proteins but we will only look at the two first. Integral proteins extend through the membrane and acts as channels for eks ions, they are also transporters, enzymes and receptors. Most important is that the electrical behavior of the cell is controlled by these proteins. Peripheral proteins  unlike integral proteins are mainly on the inside of the membrane and are temporary bound to lipids. The main tasks of the peripheral proteins is to direct and maintain both the intracellular cytoskeleton and components of the extracellular matrix. Generally the functions of the membrane proteins are: connective (membrane-cytoskeleton and extracellular matrix), transport (in and out of the cell), receptor for molecules, signal receptor (receives and transmits) and catalysis (reactions on the membrane).

Extracellular matrix...




\end{multicols}

%\appendix
%\section{Appendix}

%\begin{align}
%\left<E_+\right> = \mel{\Psi_{AB+}}{\hat{H}}{\Psi_{AB+}}\label{eq:03}\\
%\left<E_-\right> = \mel{\Psi_{AB-}}{\hat{H}}{\Psi_{AB-}}\label{eq:04}
%\end{align}

%Following the calculations from equation \ref{eq:03}
%\begin{align}
%\left<E_+\right> = \mel{\Psi_{AB+}}{\hat{H}}{\Psi_{AB+}}\\
%=\lp |c_A|^2\mel{\psi_{A}}{\hat{H}}{\psi_{A}} + |c_B|^2\mel{\psi_{B}}{\hat{H}}{\psi_{B}} + c_A^*c_B\mel{\psi_{A}}{\hat{H}}{\psi_{B}} + c_B^*c_A\mel{\psi_{B}}{\hat{H}}{\psi_{A}}\rp.
%\end{align}
%In this case the Hamiltonian is
%\begin{equation}
%\hat{H} = \hat{H}_0 - V_e + V_p
%\end{equation}
%where \(V_p\) is the potential from the other proton and \(V_e\) is the potential from the other electron.


%\onecolumngrid

%\bibliographystyle{plain}
%\bibliography{referanser} 

\end{document}

